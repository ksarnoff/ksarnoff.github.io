%%%%%%%%%%%%%%%%%%%%%%%%%%%%%%%%%%%%%%%%%
% Medium Length Professional CV
% LaTeX Template
% Version 2.0 (8/5/13)
%
% This template has been downloaded from:
% http://www.LaTeXTemplates.com
%
% Original author:
% Trey Hunner (http://www.treyhunner.com/)
%
% Important note:
% This template requires the resume.cls file to be in the same directory as the
% .tex file. The resume.cls file provides the resume style used for structuring the
% document.
%
%%%%%%%%%%%%%%%%%%%%%%%%%%%%%%%%%%%%%%%%%

%----------------------------------------------------------------------------------------
%	PACKAGES AND OTHER DOCUMENT CONFIGURATIONS
%----------------------------------------------------------------------------------------

\documentclass{resume} % Use the custom resume.cls style
\usepackage{enumitem}
\usepackage[left=1in,top=1in,right=1in,bottom=1in]{geometry} % Document margins
\usepackage{color}
\definecolor{airforceblue}{rgb}{0.20, 0.30, 1}
\definecolor{lightgrey}{rgb}{0.3,0.4,0.3}
\definecolor{linkblue}{RGB}{0, 47, 167} % {0.07, 0.17, 0.44} % {0.17,0.25,0.6} % {0.24,0.37,0.77}
\usepackage{hyperref}
\usepackage{lscape}
\usepackage{comment}
\usepackage{kpfonts}
\hypersetup{
   colorlinks=true, %set true if you want colored links
   linktoc=all,        %set to all if you want both sections and subsections linked
   linkcolor=linkblue,  %choose some color if you want links to stand out
   urlcolor  = linkblue,
   citecolor = black,
}
\name{  \LARGE Kim Sarnoff} % Your name

\address{ \href{mailto:ksarnoff@princeton.edu}{ksarnoff@princeton.edu} \\ \href{https://ksarnoff.github.io}{ksarnoff.github.io}
	\\ 646-573-8811 }

% two even smaller alternatives to \smallskip
\newskip\tinyskipsize
\tinyskipsize=1pt plus 0.5pt minus 0.5pt
%
\newcommand{\tinyskip}{\vspace{\tinyskipsize}}
%
\newskip\smallerskipsize
\smallerskipsize=1.5pt plus 0.5pt minus 0.5pt
%
\newcommand{\smallerskip}{\vspace{\smallerskipsize}}

% hskip smaller than \,
\newcommand{\nudge}{\hspace{0.08em}}
\newcommand{\datesep}{{\nudge}-{\nudge}}

\begin{document}

\vspace{0.3cm}

\begin{tabular}{@{\hspace{0cm}}llll}
	{\bf Placement Director} & Gianluca Violante & \href{mailto:violante@princeton.edu}{violante@princeton.edu} & 609-258-4003 \smallskip \\ {\bf Graduate Administrator} & Laura Hedden & \href{mailto:lhedden@princeton.edu}{lhedden@princeton.edu} & 609-258-4006
\end{tabular}

\vspace{0.3cm}

%----------------------------------------------------------------------------------------
%	CONTACT
%----------------------------------------------------------------------------------------

\begin{rSection}{Office Contact Information}

   % use \smallerskip after each line to make things fit on 2 pages
Julis Romo Rabinowitz Building \smallerskip \\
Department of Economics \smallerskip \\
Princeton University \smallerskip \\
Princeton, NJ 08544

\end{rSection}

%----------------------------------------------------------------------------------------
%	GRADUATE STUDIES
%----------------------------------------------------------------------------------------

\begin{rSection}{Graduate Studies}

{\bf Princeton University} \medskip \hfill {\em 2019{\datesep}present} \\
PhD Candidate in Economics \smallerskip \\
% Dissertation: \emph{``Essays on Belief Formation''} \smallerskip \\
Expected Completion Date: June 2026 \medskip

{\sc References}

\vspace{0.3cm}

% \begin{tabular}{@{\hspace{0cm}}ll}
% 	Professor Leeat Yariv \hspace{1cm} & Professor Ilyana Kuziemko \\ Department of Economics & Department of Economics \\ Princeton University & Princeton University \\ 609-258-4021 & 609-258-8276 \\ \href{mailto:lyariv@princeton.edu}{lyariv@princeton.edu} & \href{mailto:kuzimeko@princeton.edu}{kuziemko@princeton.edu} \bigskip \\
% 	Professor Alessandro Lizzeri \hspace{1cm} \\ Department of Economics \\ Princeton University \\ 917-756-0238 \\ \href{mailto:lizzeri@princeton.edu}{lizzeri@princeton.edu}
% \end{tabular}


\begingroup
\setlength{\tabcolsep}{8pt} % default is 6pt; increase to add space
\begin{tabular}{@{}p{0.30\textwidth} p{0.30\textwidth} p{0.30\textwidth}}
  Professor Leeat Yariv &
  Professor Ilyana Kuziemko &
  Professor Alessandro Lizzeri \\
  Department of Economics & Department of Economics & Department of Economics \\
  Princeton University & Princeton University & Princeton University \\
  609-258-4021 & 609-258-8276 & 917-756-0238 \\
  \href{mailto:lyariv@princeton.edu}{lyariv@princeton.edu} &
  \href{mailto:kuziemko@princeton.edu}{kuziemko@princeton.edu} &
  \href{mailto:lizzeri@princeton.edu}{lizzeri@princeton.edu}
\end{tabular}
\endgroup

\end{rSection}

%----------------------------------------------------------------------------------------
%	PRIOR EDUCATION
%----------------------------------------------------------------------------------------

\begin{rSection}{Prior Education}

{\bf Brown University} \medskip \hfill {\em 2012{\datesep}2016} \\
B.A. in Economics and Public Policy

\end{rSection}

%----------------------------------------------------------------------------------------
%	FIELDS
%----------------------------------------------------------------------------------------

\begin{rSection}{Fields}
	\begin{tabular}{@{\hspace{0cm}}ll}
		{\sc Primary} & Experimental, Behavioral \medskip \\ {\sc Secondary} & Labor
	\end{tabular}
\end{rSection}


%----------------------------------------------------------------------------------------
%	PUBLICATIONS
%----------------------------------------------------------------------------------------

\begin{rSection}{Publications}

\begin{enumerate}[label={},leftmargin=*]

\item
 \href{https://ksarnoff.github.io/files/papers/fsy-annualreview.pdf}{Experimental Economics: Past and Future} with Guillaume Frechette and Leeat Yariv. \textit{Annual Review of Economics}, 2022.

 \textbf{Abstract} \hspace{.1cm}
 Over the past several decades, lab experiments have offered economists a rich source of evidence on incentivized behavior. 
 In this article, we use detailed data on experimental papers to describe recent trends in the literature. 
 We also discuss various experimentation platforms and new approaches to the design and analysis of the data they generate.

\end{enumerate}

\end{rSection}

%----------------------------------------------------------------------------------------
%	WORKING PAPERS
%----------------------------------------------------------------------------------------

\begin{rSection}{Working Papers}

	\begin{enumerate}[label={},leftmargin=*]

		\item \href{https://ksarnoff.github.io/files/papers/sarnoff-jmp.pdf}{The Structure of Sequential Updating} (Job Market Paper)
		
      \textbf{Abstract} \hspace{.1cm}
      Many real-world inference problems unfold over time: employers learn about ability across tasks, consumers evaluate products through repeated use, and policymakers revise beliefs as new data arrive. Yet despite its ubiquity, research on dynamic updating has largely focused on a single implication of Bayesian reasoning: order independence. This paper experimentally tests a broader set of restrictions implied by Bayes' rule, emphasizing both order independence and the previously unexamined property of \textit{prior sufficiency}: the principle that the most recent posterior should serve as a sufficient statistic for past information. In a multi-period updating experiment with a rich set of parameters, participants repeatedly revise beliefs after receiving signals of varying strength and structure. Three main results emerge. First, only roughly a third display order dependence, overreacting to conflicting signals. Second, violations of prior sufficiency are widespread: beliefs formed sequentially tend to grow more extreme, and models assuming prior sufficiency, such as Grether (1980), fit poorly beyond the first update. Finally, the data indicate that participants process signals in aggregate, explaining prior sufficiency violations.

      \bigskip

		\item \href{https://ksarnoff.github.io/files/papers/adgs-trust.pdf}{Female Entrepreneurship and Trust in the Market} (with Nava Ashraf, Alexia Delfino, and Edward Glaeser). Revise and resumbit at \textit{Journal of Political Economy}.
		
      \textbf{Abstract} \hspace{.1cm}
      Commerce requires trust, but trust is difficult when one group can expropriate another due to differences in power. 
      This can lead the weaker group to self-segregate into industries and activities; female-led businesses, for example, tend to be small and clustered in a small number of industries where collaborators are also female.
      We present a model which relates this economic segregation to rule of law, and predicts that female trust depends on the protective preferences of adjudicators in weak rule of law environments. 
      We then show that effective dispute negotiation in Lusaka, Zambia, especially as administered by ``market chiefs,'' enables trusting behavior by female entrepreneurs, both in cross-section correlations and in two artefactual field experiments. 
      Such trust generates increased economic returns. 
      We find considerable heterogeneity across market chiefs in their preferences for protecting women, and that female entrepreneurs are more likely to want to reveal their gender with chiefs who are more likely to favor women. 

      \bigskip

		\item \href{https://ksarnoff.github.io/files/papers/ss-misconduct.pdf}{Misconduct in Organizations} (with Hassan Sayed)
	
      \textbf{Abstract} \hspace{.1cm} 
      We study how policies that disincentivize misconduct in organizations can generate changes in abusers' behaviors that negatively impact victims. We describe a model where ``managers'' choose to commit harmful actions of varying intensities against ``employees,'' who can report these actions as ``misconduct.'' 
      We show that when the marginal disutility from managers' actions is particularly small, increasing the ease of reporting misconduct, the severity of punishment for managers, or the efficacy of investigation technology may in fact harm employees. These policies may motivate managers to commit harmful actions that employees do not want to report or induce managers to opt out of interacting with employees altogether. We provide a dynamic extension where reports generate precedents for the organization and employees, showing that the model converges to a steady state where employees are worse off than initially and harmful actions are never punished.

	\end{enumerate}

\end{rSection}

\begin{rSection}{Work in Progress}

	\begin{enumerate}[label={},leftmargin=*]

		\item Dynamic Updating about Menus
		
		\textbf{Abstract}\hspace{.1cm}
      In many environments, updating occurs in small increments over time and agents juggle belief revisions across a menu of options: for example, teachers assess multiple students over a long period of interaction. This paper uses an array of online experiments to explore how people respond to uni-dimensional and multi-dimensional information over various horizons. I develop an econometric approach for estimating the weights individuals place on their prior beliefs and on the signals they receive over time. I use this approach to document several patterns. First, participants underweigh the prior but overweigh signals. Second, the weight placed on signals changes over time. That time dependence leads to substantial sequencing effects and an apparent recency bias that becomes more pronounced over time. Third, receiving simultaneous information on multiple uncertain outcomes leads to important ``grouping'' effects: even when the initial priors over outcomes differ, posteriors converge over time. 

	\end{enumerate}

\end{rSection}

%----------------------------------------------------------------------------------------
%	RESEARCH EXPERIENCE
%----------------------------------------------------------------------------------------

\begin{rSection}{Research Experience}
\begin{tabular}{@{\hspace{0cm}}ll}
2016{\datesep}2018 & Research Assistant to Prof. Nava Ashraf (London School of Economics) \\
\end{tabular}
\end{rSection}

%----------------------------------------------------------------------------------------
%	TEACHING EXPERIENCE
%----------------------------------------------------------------------------------------

\begin{rSection}{Teaching}
\begin{tabular}{@{\hspace{0cm}}ll}
{\it Princeton} & ECO 300: Undergraduate Microeconomic Theory (TA, Fall 2022) \smallskip \\
& SPI 511C: Advanced Microeconomic Analysis for Policymakers (TA, Fall 2024) \medskip \\
\end{tabular}
\end{rSection}

%----------------------------------------------------------------------------------------
%	PROFESSIONAL ACTIVITIES
%----------------------------------------------------------------------------------------

\begin{rSection}{Professional Activities}

\textbf{Presentations and Seminars}

\begin{tabular}{lp{14cm}}
2025 & Behavioral and Experimental Economists of the Mid-Atlantic (BEEMA) Meeting \smallskip \\
2024 & Economic Science Association (ESA) Meeting \smallskip\\
\end{tabular}

{ \bf Refereeing}

\begin{tabular}{lp{14cm}}
   \emph{American Economic Journal: Applied}, \emph{American Economic Journal: Microeconomics}, \emph{Econometrica}
\end{tabular}

\end{rSection}

%\clearpage
%----------------------------------------------------------------------------------------
%	AWARDS
%----------------------------------------------------------------------------------------

\begin{rSection}{Honors, Scholarships, Fellowships, and Grants}

Hamid Biglari *87 Behavioral Science Fellowship, Princeton University \smallskip \hfill{\em 2024} \\
NSF Graduate Research Fellowship \smallskip \hfill {\em 2020}\\
J. Edward Lundy *40 Fellowship for Economics, Princeton University \smallskip \hfill {\em 2019} \\
Noah Krieger Prize for Academic Excellence, Brown University \smallskip \hfill {\em 2016} \\
Brown/Tufts/Lifespan Center for AIDS Research Summer Intern Award  \smallskip \hfill {\em 2015}  \\
Happy and John Hazen White, Sr. Internship Award, Brown University  \smallskip \hfill {\em 2015} \\
Phi Beta Kappa, Junior Class Inductee, Brown University \hfill  \smallskip \hfill {\em 2015}

\end{rSection}

%----------------------------------------------------------------------------------------
%	LANGUAGES
%----------------------------------------------------------------------------------------

\begin{rSection}{Languages}
English (native)
\end{rSection}

% \vspace{0.5cm}

% \hfill {\em Last updated: September 2024}

\end{document}
% English (native)